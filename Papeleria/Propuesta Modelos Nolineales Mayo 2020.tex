\documentclass[11pt]{article}
\usepackage[english]{babel}
\usepackage{float}
%\selectlanguage{spanish}
\usepackage[utf8]{inputenc}
\usepackage[toc,page]{appendix}
\usepackage{amsmath}
\usepackage{amsfonts}
\usepackage{amssymb}
\usepackage{graphicx}
\usepackage{fancyvrb}
%\usepackage{cite}
\usepackage{geometry}
\usepackage[numbers]{natbib}
\usepackage{verbatim}
\newcommand{\bc}{\begin{center}}
\newcommand{\ec}{\end{center}}
\geometry{left=2.5cm,right=2.5cm,top=3cm,bottom=3cm,headheight=3mm}

\title{\textbf{COVID-19 y Modelos Nolineales\\ (PLAN DE TRABAJO)}}
\author{(Nombre de Estudiante)
\\Universidad Aut\'onoma de Nuevo Le\'on (UANL)}

\date{}	
\begin{document}
	\maketitle

\begin{center}
\abstractname{}\\
\end{center}
Presentamos una comparaci\'on del desempe\~no de varios modelos nolineales
para ajustar datos de crecimiento de la epidemia COVID-19
en varios pa\'ises. \\
\noindent \textbf{Keywords}: Epidemiolog\'ia, Gompertz, Normal, Log\'istica,
Sigmoide

\tableofcontents



\underline{\hspace{15cm}}


\section{Objetivos}  \label{Obj}

El crecimiento de la pandemia COVID-19,
en diferentes pa\'ises presenta diferentes comportamientos, habiendo
diferencias significativas en cuanto a las velocidades de crecimiento,
tiempos de arranque y mesetas. Sin embargo, en t\'erminos muy generales,
se distinguen varias fases comunes tales como, un periodo de arranque
lento, una fase de crecimiento acelerado, algunas veces con tasas
exponenciales, un periodo de crecimiento desacelerado y finalmente
llegando a una  meseta con crecimiento nulo. Estas caracter\'isticas
comunes pueden modelarse mediante curvas sigmoidales.

\medskip

El objetivo del presente trabajo es presentar una comparaci\'on
de diferentes modelos nolineales que se han desarrollado en la literatura,
particularmente para COVID-19. Los diferentes aspectos a desarrollar
incluyen, pero no se limitan a:
\begin{itemize}
\item Caracterizaci\'on de los diferentes par\'ametros que definen a cada modelo.
\item C\'odigo $R$ para el ajuste de modelos, as\'i como el c\'alculo
de medidades de incertidumbre asociados al proceso de estimaci\'on.
\item Adem\'as de los modelos Gompertz, Normal, Log\'istico, ?`Qu\'e
otros modelos se han estudiado?.
\item Los modelos de crecimiento tienen asociados, t\'ipicamente,
una ecuaci\'on o ecuaciones diferenciales de las cuales son ellos son
soluciones. Presentar estas ecuaciones.
\item Mostrar la incorporaci\'on de covariables en el ajuste de los modelos.
\end{itemize}



\section{Entregables}  \label{Entregables}

\begin{itemize}
\item Reporte en latex
\item C\'odigo $R$ comentado
\item Presentaci\'on beamer con los resultados del estudio.
\end{itemize}








\section{Bibliograf\'ia}  \label{Bibliografia}


\begin{itemize}
\item Huet, S. \textit{et al.} (2004). Statistical Tools for Nonlinear Regression. Springer.  

\item IHME COVID-19 forecasting team (2020). Forecasting COVID-19 impact on hospital bed-days, ICU-days, ventilator days
and deaths by US state in the next 4 months. Report

\item Prats, C. \textit{et al.} (2020) Analysis and prediction of COVID-19 for
different regions and countries. Daily report 27-03-2020. UPC, BioComSC, CMCiB, 
IGTP.  
\end{itemize}








\end{document}