\documentclass[11pt]{article}
\usepackage[english]{babel}
\usepackage{float}
%\selectlanguage{spanish}
\usepackage[utf8]{inputenc}
\usepackage[toc,page]{appendix}
\usepackage{amsmath}
\usepackage{amsfonts}
\usepackage{amssymb}
\usepackage{graphicx}
\usepackage{fancyvrb}
%\usepackage{cite}
\usepackage{geometry}
\usepackage[numbers]{natbib}
\usepackage{verbatim}
\newcommand{\bc}{\begin{center}}
\newcommand{\ec}{\end{center}}
\geometry{left=2.5cm,right=2.5cm,top=3cm,bottom=3cm,headheight=3mm}

\title{\textbf{COVID-19 y Modelos Epidemiológicos Compartimentales\\ (PLAN DE TRABAJO)}}
\author{Karen Eunice Oviedo Garza
\\Universidad Aut\'onoma de Nuevo Le\'on (UANL)}

\date{}	
\begin{document}
	\maketitle

\begin{center}
\abstractname{}\\
\end{center}
Estudiamos algunos modelos epidemiológicos compartimentales básicos para describir la propagación de la
epidemia del COVID-19 en población mexicana.\\
\noindent \textbf{Keywords}: Modelos compartimentales, SIR, Epidemiología, Ecuaciones diferenciales.

\renewcommand\contentsname{Contenido}
\tableofcontents



\underline{\hspace{15cm}}


\section{Objetivos}  \label{Obj}

Para poder entender y en su momento controlar o mitigar los efectos
de una pandemia como COVID-19, es importante contar con modelos que nos perimitan 
predecir como se esparce dicha infección, el número total de infectados y
la duración de la epidemia; esto, bajo distintos escenarios que reflejen características de
la enfermedad. Los modelos compartimentales se presentan como un marco de modelación que puede dar
respuesta a las cuestiones anteriores y que en la práctica han demostrado ser confiables y de utilidad.


\medskip

El objetivo del presente trabajo es estudiar algunos modelos compartimentales básicos, como
el modelo SIR, y utilizarlos para analizar algunos ejemplos de epidemias, haciendo énfasis
en el COVID-19. Nos enfocaremos en los supuestos bajo los cuales funciona el modelo y sus 
posibles limitaciones. Los diferentes aspectos a desarrollar incluyen, pero no se limitan a:

\begin{itemize}
\item Modelos SIS, SIR y epidemiológico de Kermack–McKendrick.
\item C\'odigo $R$ para el ajuste de los modelos a partir de datos reales.
\item El uso de modelos comparimentales en el caso del COVID-19.
\end{itemize}



\section{Entregables}  \label{Entregables}

\begin{itemize}
\item Reporte en latex.
\item C\'odigo $R$ comentado.
\item Presentaci\'on beamer con los resultados del estudio.
\end{itemize}





\section{Bibliograf\'ia}  \label{Bibliografia}

Algunas referencias que convendría tener en cuenta son las siguientes:

\begin{itemize}
\item Arino, Julien, and Stéphanie Portet. "A simple model for COVID-19." Infectious Disease Modelling (2020).

\item  Brauer, Fred, Carlos Castillo-Chavez, and Zhilan Feng. Mathematical Models in Epidemiology. Springer New York, 2019..

\item	Giordano, Giulia, et al. "Modelling the COVID-19 epidemic and implementation of population-wide interventions in Italy." Nature Medicine (2020): 1-6.

\item Ndairou, Faical, et al. "Mathematical modeling of COVID-19 transmission dynamics with a case study of wuhan." Chaos, Solitons \& Fractals (2020): 109846.


\end{itemize}

\bigskip

\bigskip

\bigskip

\bigskip

\bigskip

\underline{\hspace{6cm}}

\hspace{.5cm} Karen Eunice Oviedo Garza





\end{document}